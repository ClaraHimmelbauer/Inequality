%----------------------------------------------------------------------------------------
%    PACKAGES AND THEMES
%----------------------------------------------------------------------------------------

\documentclass[aspectratio=169,xcolor=dvipsnames]{beamer}
\usetheme{SimplePlus}

\usepackage[utf8]{inputenc}		% defines file's character encoding
\usepackage[ngerman]{babel}     % babel system, adjust the language of the content

\usepackage{hyperref}
\usepackage{graphicx} % Allows including images
\usepackage{xcolor}

% for the table
\usepackage{booktabs}
\usepackage{multirow}
\usepackage{tabularx}
\usepackage{longtable}
\usepackage{adjustbox}

%frame
\usepackage{mdframed}

%\usepackage[backend=bibtex]{biblatex} % Use biblatex with BibTeX
%\addbibresource{reference.bib}       % Your .bib file

%----------------------------------------------------------------------------------------
% TITLE PAGE
%----------------------------------------------------------------------------------------

\title{Ungleichheit}
\subtitle{Ringvorlesung Innsbruck}

\author{Clara Himmelbauer}

\institute
{
    Research Institute Economics of Inequality (INEQ) \\
    WU Vienna % Your institution for the title page
}
\date{December 3rd, 2025} % Date, can be changed to a custom date

%----------------------------------------------------------------------------------------
%    PRESENTATION SLIDES
%----------------------------------------------------------------------------------------

\begin{document}

\begin{frame}
    % Print the title page as the first slide
    \titlepage
\end{frame}

% \begin{frame}{Project Overview}
%     This work is part of a project together with Lukas Heck. We want to see how outsourcing of childcare interacts with social class. I present the quanti results but we are also conducting a qualitative analysis.
% \end{frame}

\begin{frame}{Plan für heute}

    \tableofcontents
\end{frame}

%------------------------------------------------
\section{Dimensionen von Ungleichheit}
%------------------------------------------------

\begin{frame}[plain,c]
    \begin{center}
    \usebeamerfont{title}
    \color{DarkBlue}
        Dimensionen von Ungleichheit
    \end{center}
\end{frame}


\begin{frame}{Dimensionen von Ungleichheit}
    Ungleichheit von was?
    \pause
    \begin{itemize}
        \item Einkommen, Vermögen
        \item Bildung, Zeit, Klassen, Konsum, ...
        \item Outcomes vs. Chancen
    \end{itemize}

    \pause
    \vspace{0.5cm}
    Ungleichheit zwischen wem?
    \pause
    \begin{itemize}
        \item Individuen
        \item Haushalte
        \item Staaten
    \end{itemize}
\end{frame}

\begin{frame}{Einkommensverteilung Österreich und global}
    
\end{frame}

\begin{frame}{Vermögensverteilung Österreich und global}
    
\end{frame}

\begin{frame}[label = zeitverwendung]{Verteilung von Arbeit}

    \begin{figure}
        \centering
        \includegraphics[width=0.8\linewidth]{figures/verteilung_allgemein/zve_arbeitszeit.png}
        \label{fig:zve_verteilung-arbeit}
    \end{figure}

    \hyperlink{zeitverwendung_erwachsene}{\beamerskipbutton{Verteilung von Arbeitszeit 25-59-Jährige}}
\end{frame}

%------------------------------------------------
\section{Arm und Reich}
%------------------------------------------------

\begin{frame}[plain,c]
    \begin{center}
    \usebeamerfont{title}
    \color{DarkBlue}
        Arm und Reich
    \end{center}
\end{frame}


\begin{frame}{Diskussion}
    \begin{itemize}
        \item Wann ist man arm?
        \item Wann ist man reich?
    \end{itemize}
\end{frame}

\begin{frame}[label = armutsmessung]{Armut}
    \begin{columns}
        \begin{column}{.3\textwidth}
            \begin{block}{Absolute Armutsbetroffenheit}
                \begin{itemize}
                    \item Extreme Armutsschwelle: weniger als \$3 pro Person und Tag (in Kaufkraftparitäten)
                    \item Weltbank-Definition
                \end{itemize}
            \end{block}
        \end{column}

        \pause
        \begin{column}{.3\textwidth}
            \begin{block}{Relative Armutsgefährdung}
                \begin{itemize}
                    \item Haushaltseinkommen weniger als 60\% des (nationalen) äquivalisierten Median-Haushaltseinkommens
                    \item OECD-Definition, Messung mittels EU-SILC
                \end{itemize}   
            \end{block}            
        \end{column}

        \pause
        \begin{column}{.3\textwidth}
            \begin{block}{Erhebliche materielle und soziale Deprivation}
                \begin{itemize}
                    \item Nichterfüllung der Grundbedürfnisse in mindestens 7 von 13 Merkmalen
                    \item Messung mittels EU-SILC
                \end{itemize}    
            \end{block}

        \end{column}

    \end{columns}
\end{frame}

\begin{frame}{Menschen in absoluter Armut}
        
    \begin{figure}
        \centering
        \includegraphics[width=0.675\linewidth]{figures/armut/wordbank_poverty-rate-by-country-2025.png}
        \label{fig:weltbank_poverty-countries}
    \end{figure}

    Weltweit leben über 600 Millionen (7,9\%) Menschen in extremer Armut
\end{frame}

\begin{frame}{Armut in Österreich}
    \begin{figure}
        \centering
        \includegraphics[width=0.8\linewidth]{figures/armut/statA_armutsgefaehrdung-oesterreich.png}
        \label{fig:armutsgefaehrdung-oesterreich}
    \end{figure}
\end{frame}

\begin{frame}{Reichtum}
    Reichtumsbetroffenheit?
    Wie misst man Reichtum?
\end{frame}

%------------------------------------------------
\section{Auswirkungen von Ungleichheit}
%------------------------------------------------

\begin{frame}[plain,c]
    \begin{center}
    \usebeamerfont{title}
    \color{DarkBlue}
        Auswirkungen von Ungleichheit
    \end{center}
\end{frame}


\begin{frame}{Probleme durch Ungleichheit}
    \begin{itemize}
        \item Wählt in Kleingruppen eine Dimension, in der Ungleichheit zum Problem werden könnte.
        \item Diskutiert wieso Ungleichheit hier ein Problem ist
        \item Welche politischen Maßnahmen könnte man setzen?
    \end{itemize}

    \vspace{0.5cm}
    \pause
    \begin{itemize}
        \item Demokratie \& Ungleichheit
        \item Klimawandel \& Ungleichheit
        \item Auswirkungen auf Chancengleichheit und soziale Mobilität
        \item Intersektionale Perspektive auf Ungleichheit
    \end{itemize}
\end{frame}

\begin{frame}{Demokratie}
    politische einflussnahme: lobbying, wahlkampfspenden, eigene parteien, androhung ins ausland abzuwandern
    medien, think-tanks
    korruption
    vertrauen in politische institutionen sinkt
    armut führt zu geringerer politischer teilhabe
    arme personen bei nichtwahlberechtigten überrepräsentiert.
\end{frame}

\begin{frame}{Klimawandel}
    wer stößt co2 aus - personen in der einkommensverteilung, länder
    wer ist von den folgen des klimawandels betroffen - personen in der einkommensverteilung, länder
\end{frame}

\begin{frame}{Soziale Mobilität}
    was ist soziale mobilität:
    humankapital, netzwerke, mehr förderung, mehr geld für gute schulen, kindergärten, außerschulische aktivitäten, förderung
    gesündere lebensbedingungen (essen, wohnsituation). raum für sich, ruhe
\end{frame}

\begin{frame}{Intersektional}
    wer hat geld, wer hat es nicht
    race, class, gender, alter, migrationshintergrund
    ungleichheiten verstärken sich gegenseitig
    erfahrungen von reicher frau sind nicht gleich wie erfahrngen von armer frau. erfahrungen einer österreichierin mit ungleichheit sind nicht gleich wie jene einer migrantin
\end{frame}

%------------------------------------------------
\section{Was ist gerecht?}
%------------------------------------------------

\begin{frame}[plain,c]
    \begin{center}
    \usebeamerfont{title}
    \color{DarkBlue}
        Was ist gerecht?
    \end{center}
\end{frame}


\begin{frame}{Was ist gerecht?}
    \begin{itemize}
        \item Wann ist Ungleichheit gerecht?
        \item Was ist das richtige Maß an Ungleichheit?

        \vspace{0.5cm}
        \pause
        \item Gibt es Armut ohne Reichtum?
        \item Gibt es Reichtum ohne Armut?
    \end{itemize}

    Links zum Appendix, wie Leute über Ungleichheit denken
\end{frame}

%------------------------------------------------
\section{Ceterum Censeo}
%------------------------------------------------

\begin{frame}[plain,c]
    \begin{center}
    \usebeamerfont{title}
    \color{DarkBlue}
        Wrap-up
    \end{center}
\end{frame}

\begin{frame}{Ceterum Censeo - Punkte, die mir noch wichtig sind}
    \begin{itemize}
        \item Ungleichheit kann politisch bekämpft werden. Ungleichheit war am geringsten in den 70ern, aufgrund starker Sozialstaaten.

        \pause
        \vspace{0.5cm}
        \item Wo stehst du selbst?
        \item Vor allem: Wo bist du im Vergleich zu anderen privilegiert? Denke nicht nur an Einkommen und Vermögen.
        \item Wie schlägt sich das nieder?
    \end{itemize}
\end{frame}

\begin{frame}{Ressourcen}
    \begin{itemize}
        \item World Inequality Report
        \item World Inequality Databank: https://wid.world/
    \end{itemize}
\end{frame}

\begin{frame}{Feedback}
    \begin{center}
        ?`Fragen?
    \end{center}

    \vspace{0.5cm}
    \pause 
    Feedback
    \begin{itemize}
        \item Was hat euch gefallen? Was hat euch nicht gefallen? Was war so naja und könnte ich besser machen?
        \item Was nehmt ihr mit?
        \item Worüber hättet ihr gerne mehr gehört?
    \end{itemize}

    \vspace{0.5cm}
    \pause
    \begin{center}
        Kontakt: clara.himmelbuaer@wu.ac.at
    \end{center}
\end{frame}


% \begin{frame}[allowframebreaks]{References}
%     \nocite{*} 
%     \footnotesize
%     \bibliography{ref.bib}
%     \bibliographystyle{apalike}
% \end{frame}

%------------------------------------------------
\section*{Appendix}
%------------------------------------------------

\begin{frame}[plain,c]
    \begin{center}
    \usebeamerfont{title}
    \color{DarkBlue}
        Appendix
    \end{center}
\end{frame}


\begin{frame}{Wie denken Leute über Ungleichheit?}
    
\end{frame}

\begin{frame}{Messung von Ungleichheit}
    
\end{frame}


\begin{frame}[label = zeitverwendung_erwachsene]{Verteilung von Arbeit}

    \begin{figure}
        \centering
        \includegraphics[width=0.8\linewidth]{figures/verteilung_allgemein/zve_arbeitszeit_25-59.png}
        \label{fig:zve_verteilung-arbeit_25-59}
    \end{figure}

    \hyperlink{zeitverwendung}{\beamerskipbutton{Verteilung von Arbeitszeit Zeitverwendungserhebung}}
\end{frame}


\end{document}