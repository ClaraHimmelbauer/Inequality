%----------------------------------------------------------------------------------------
%    PACKAGES AND THEMES
%----------------------------------------------------------------------------------------

\documentclass[aspectratio=169,xcolor=dvipsnames]{beamer}
\usetheme{SimplePlus}

\usepackage[utf8]{inputenc}		% defines file's character encoding
\usepackage[ngerman]{babel}     % babel system, adjust the language of the content

\usepackage{hyperref}
\usepackage{graphicx} % Allows including images
\usepackage{xcolor}

% for the table
\usepackage{booktabs}
\usepackage{multirow}
\usepackage{tabularx}
\usepackage{longtable}
\usepackage{adjustbox}

%frame
\usepackage{mdframed}

%\usepackage[backend=bibtex]{biblatex} % Use biblatex with BibTeX
%\addbibresource{reference.bib}       % Your .bib file

%----------------------------------------------------------------------------------------
% TITLE PAGE
%----------------------------------------------------------------------------------------

\title{Ungleichheit}
\subtitle{Zukunftsfähiges Wirtschaften in Zeiten multipler Krisen}

\author{Clara Himmelbauer}

\institute
{
    Research Institute Economics of Inequality (INEQ) \\
    WU Vienna % Your institution for the title page
}
\date{December 3rd, 2025} % Date, can be changed to a custom date

%----------------------------------------------------------------------------------------
%    PRESENTATION SLIDES
%----------------------------------------------------------------------------------------

\begin{document}

\begin{frame}
    % Print the title page as the first slide
    \titlepage
\end{frame}

% \begin{frame}{Project Overview}
%     This work is part of a project together with Lukas Heck. We want to see how outsourcing of childcare interacts with social class. I present the quanti results but we are also conducting a qualitative analysis.
% \end{frame}

\begin{frame}{Plan für heute}

    \tableofcontents
\end{frame}

%------------------------------------------------
\section{Dimensionen von Ungleichheit}
%------------------------------------------------

\begin{frame}[plain,c]
    \begin{center}
    \usebeamerfont{title}
    \color{DarkBlue}
        Dimensionen von Ungleichheit
    \end{center}
\end{frame}


\begin{frame}{Dimensionen von Ungleichheit}

    \begin{columns}
        \begin{column}{.45\textwidth}
            \begin{block}{Ungleichheit von was?}
                \pause
                \begin{itemize}
                    \item Einkommen, Vermögen
                    \item Bildung, Zeit, Klassen, Konsum, ...
                    \item Outcomes vs. Chancen
                \end{itemize}
            \end{block}
        \end{column}
        
        \pause
        \begin{column}{.45\textwidth}
            \begin{block}{Ungleichheit von wem?}
                \pause
                \begin{itemize}
                    \item Individuen
                    \item Haushalte
                    \item Staaten
                \end{itemize}
            \end{block}
        \end{column}

    \end{columns}
\end{frame}

\begin{frame}{Globale Einkommens- und Vermögensverteilung}
    \begin{figure}
        \centering
        \includegraphics[width=0.8\linewidth]{figures/verteilung_allgemein/wir_global-income-wealth-inequality.png}
        \label{fig:wir_global-income-wealth-ineq}
    \end{figure}
\end{frame}

\begin{frame}{Globale Einkommensverteilung: Within \& Between-Country}
    \begin{figure}
        \centering
        \includegraphics[width=0.8\linewidth]{figures/verteilung_allgemein/wir_within-between-country-ineq.png}
        \label{fig:wir_within-between-ineq}
    \end{figure}
\end{frame}

\begin{frame}{Globales Einkommenswachstum}
    \begin{figure}
        \centering
        \includegraphics[width=0.65\linewidth]{figures/verteilung_allgemein/elephant-graph.png}
        \label{fig:elephant-graph}
    \end{figure}
\end{frame}

\begin{frame}[label = ekverteilung_at]{Nationale Einkommensverteilung}

    \begin{columns}
        \begin{column}{0.7\textwidth}
        \begin{figure}
            \centering
            \includegraphics[width=1\linewidth]{figures/verteilung_allgemein/ak_einkommensverteilung-oesterreich.png}
            \label{fig:ak_einkommensverteilung}
        \end{figure}
        \end{column}

        \begin{column}{0.25\textwidth}
            \hyperlink{ekverteilung_alternativ}{\beamerskipbutton{Alternative Darstellung}}
            \hyperlink{gender-pay-gap}{\beamerskipbutton{Gender-Pay Gap}}
        \end{column}
    \end{columns}

\end{frame}


\begin{frame}[label = vermoegensverteilung_at]{Nationale Vermögensverteilung}

    \begin{columns}
        \begin{column}{0.7\textwidth}
        \begin{figure}
            \centering
            \includegraphics[width=1\linewidth]{figures/verteilung_allgemein/hfcs-2010_bruttovermoegensverteilung.png}
            \label{fig:hfcs_vermoegensverteilung}
        \end{figure}
        \end{column}

        \begin{column}{0.25\textwidth}
            \hyperlink{vermoegensverteilung-alternativ}{\beamerskipbutton{Alternative Darstellung}}
            \hyperlink{vermoegensverteilung-kumulativ}{\beamerskipbutton{Kumulierte Vermögensverteilung}}
        \end{column}
    \end{columns}

\end{frame}
\begin{frame}[label = zeitverwendung]{Verteilung von Arbeit}

    \begin{figure}
        \centering
        \includegraphics[width=0.8\linewidth]{figures/verteilung_allgemein/zve_arbeitszeit.png}
        \label{fig:zve_verteilung-arbeit}
    \end{figure}

    \hyperlink{zeitverwendung_erwachsene}{\beamerskipbutton{Verteilung von Arbeitszeit 25-59-Jährige}}
    \hyperlink{gender-pay-gap}{\beamerskipbutton{Gender-Pay-Gap}}
\end{frame}

%------------------------------------------------
\section{Arm und Reich}
%------------------------------------------------

\begin{frame}[plain,c]
    \begin{center}
    \usebeamerfont{title}
    \color{DarkBlue}
        Arm und Reich
    \end{center}
\end{frame}


\begin{frame}{Diskussion}
    \begin{itemize}
        \item Wann ist man arm?
        \item Wann ist man reich?
    \end{itemize}
\end{frame}

\begin{frame}[label = armutsmessung]{Armut}
    \begin{columns}
        \begin{column}{.3\textwidth}
            \begin{block}{Absolute Armutsbetroffenheit}
                \begin{itemize}
                    \item Extreme Armutsschwelle: weniger als \$3 pro Person und Tag (in Kaufkraftparitäten)
                    \item Weltbank-Definition
                \end{itemize}
            \end{block}
        \end{column}

        \pause
        \begin{column}{.3\textwidth}
            \begin{block}{Relative Armutsgefährdung}
                \begin{itemize}
                    \item Haushaltseinkommen weniger als 60\% des (nationalen) äquivalisierten Median-Netto-Haushalts-Einkommens
                    \item OECD-Definition, Messung mittels EU-SILC
                \end{itemize}   
            \end{block}            
        \end{column}

        \pause
        \begin{column}{.3\textwidth}
            \begin{block}{Erhebliche materielle und soziale Deprivation}
                \begin{itemize}
                    \item Nichterfüllung der Grundbedürfnisse in mindestens 7 von 13 Merkmalen
                    \item Messung mittels EU-SILC
                \end{itemize}    
            \end{block}

        \end{column}

    \end{columns}
\end{frame}

\begin{frame}{Menschen in absoluter Armut}
        
    \begin{figure}
        \centering
        \includegraphics[width=0.675\linewidth]{figures/armut/wordbank_poverty-rate-by-country-2025.png}
        \label{fig:weltbank_poverty-countries}
    \end{figure}

    Weltweit leben über 600 Millionen (7,9\%) Menschen in extremer Armut
\end{frame}

\begin{frame}{Armut in Österreich}
    \begin{figure}
        \centering
        \includegraphics[width=0.8\linewidth]{figures/armut/statA_armutsgefaehrdung-oesterreich.png}
        \label{fig:armutsgefaehrdung-oesterreich}
    \end{figure}
\end{frame}

\begin{frame}{Reichtum}
    \begin{itemize}
        \item Reichtumsgrenze bei 300\% des äquivalisierten Median-Netto-Haushalts-Einkommens?
        \item Was ist mit Vermögen?
    \end{itemize}

    \pause
    \begin{columns}
        \begin{column}{0.55\textwidth}
        \begin{itemize}
            \item Wir haben keine guten Daten zu Reichtum
            \item Annäherung mittels HFCS \& Reichenlisten (Forbes, Trend)
        \end{itemize}
        \end{column}
        
        \begin{column}{0.4\textwidth}
            \begin{figure}
                \begin{flushright}
                    \includegraphics[width=1\linewidth]{figures/memes/wealth-tax-for-data.jpg}
                \end{flushright}
                \label{fig:wealth-tax-for-data}
            \end{figure}
        \end{column}
    \end{columns}
    
\end{frame}

%------------------------------------------------
\section{Auswirkungen von Ungleichheit}
%------------------------------------------------

\begin{frame}[plain,c]
    \begin{center}
    \usebeamerfont{title}
    \color{DarkBlue}
        Auswirkungen von Ungleichheit
    \end{center}
\end{frame}


\begin{frame}{Probleme durch Ungleichheit}
    \begin{itemize}
        \item Wählt in Kleingruppen eine Dimension von Ungleichheit.
        \item Wieso ist Ungleichheit hier ein Problem?
        \item Welche politischen Maßnahmen könnte man setzen?
    \end{itemize}

    \vspace{0.5cm}
    \pause
    \begin{itemize}
        \item Demokratie \& Ungleichheit
        \item Klimawandel \& Ungleichheit
        \item Auswirkungen auf Chancengleichheit und soziale Mobilität
        \item Intersektionale Perspektive auf Ungleichheit
    \end{itemize}
\end{frame}

\begin{frame}[label = demokratie_einflussnahme]{Demokratie: Politische Einflussnahme}
    Politsche Einflussnahme über:
        \begin{itemize}
            \item Lobbying \hyperlink{lobbying}{\beamerskipbutton{Lobbying}}
            \item Parteien finanzieren oder gründen \hyperlink{parteispenden}{\beamerskipbutton{Parteispenden}}
            \item Überrepräsentation in politischen Ämtern \hyperlink{ueberrepraesentation}{\beamerskipbutton{Überrepräsentation}}
            \item Politische Einflussnahme über Medien \hyperlink{medien}{\beamerskipbutton{Medienlandschaft Österreich}}
            \item Finanzierung von Think-Tanks
            \item Korruption: Direkte Zahlungen an Politiker
            \item Drehtür-Effekt, Postenschacher
        \end{itemize}

    \hyperlink{meme-korruption}{\beamerskipbutton{Cartoon Korruption \& Lobbying}}
\end{frame}

\begin{frame}[label = demokratie_teilhabe]{Demokratie: sonstige negative Effekte von Ungleichheit}
    Vertrauen sinkt durch (Wirtschafts-)Politik für die Reichen
    \begin{itemize}
        \item Vertrauen in politische Institutionen
        \item Vertrauen in Demokratie als Staatsform (an der alle teilhaben können)
    \end{itemize}

    \pause
    \vspace{0.5cm}
    Geringere politische Teilhabe
    \begin{itemize}
        \item Armut korrelliert negativ mit politischer Teilhabe
        \item Armutsgefährdete bei Nichtwahlberechtigten überrepräsentiert \hyperlink{nichtwahlberechtigte}{\beamerskipbutton{Nichtwahlberechtigte}}
    \end{itemize}
\end{frame}

\begin{frame}[label = klimawandel]{Klimawandel}
    Reiche stoßen mehr CO2 aus \hyperlink{memes-klimawandel}{\beamerskipbutton{Memes}}
    \begin{itemize}
        \item Globaler Norden (= reiche Länder) emittieren mehr \hyperlink{percapita_gdp-co2}{\beamerskipbutton{BIP und CO2 Emissionen pro Kopf}}
        \item Reiche Menschen emittieren mehr \hyperlink{co2-einkommensverteilung}{\beamerskipbutton{Einkommensverteilung und Emissionen}} {\scriptsize \href{https://de.linkedin.com/posts/marko-kovic_ein-mann-der-soeben-mit-einem-privatjet-activity-7084645559055069184-4y1S}{Strg+F Menschen auf Sylt}}
    \end{itemize}

    \vspace{0.5cm}
    \pause
    Arme sind stärker vom Klimawandel betroffen
    \begin{itemize}
        \item Arme Länder stärker betroffen \hyperlink{climate-risk}{\beamerskipbutton{Climate Risk Index}}
        \item Weniger Möglichkeiten, die Folgen des Klimawandels abzufedern (arme Personen + arme Länder)
    \end{itemize}

    \vspace{0.5cm}
    \pause
    Maßnahmen
    \begin{itemize}
        \item Öffentliche Infrastruktur für nachhaltige Mobilität
        \item Soziale Sicherung: Schutz vor Energiearmut, Sanierungspolitik
        \item CO2-Steuer, Carbon Border Adjustment Mechanism
        \item Klimafinanzierung für arme Länder, CO2-Budgets
        \item Demokratische Teilhabe, Kontrolle, internationale Standards, ...
    \end{itemize}
\end{frame}

\begin{frame}{Soziale Mobilität}
    was ist soziale mobilität:
    humankapital, netzwerke, mehr förderung, mehr geld für gute schulen, kindergärten, außerschulische aktivitäten, förderung
    gesündere lebensbedingungen (essen, wohnsituation). raum für sich, ruhe
\end{frame}

\begin{frame}{Intersektional}
    wer hat geld, wer hat es nicht
    race, class, gender, alter, migrationshintergrund
    ungleichheiten verstärken sich gegenseitig
    erfahrungen von reicher frau sind nicht gleich wie erfahrngen von armer frau. erfahrungen einer österreichierin mit ungleichheit sind nicht gleich wie jene einer migrantin
\end{frame}

%------------------------------------------------
\section{Was ist gerecht?}
%------------------------------------------------

\begin{frame}[plain,c]
    \begin{center}
    \usebeamerfont{title}
    \color{DarkBlue}
        Was ist gerecht?
    \end{center}
\end{frame}


\begin{frame}{Was ist gerecht?}
    \begin{itemize}
        \item Wann ist Ungleichheit gerecht?
        \item Was ist das richtige Maß an Ungleichheit?

        \vspace{0.5cm}
        \pause
        \item Gibt es Armut ohne Reichtum?
        \item Gibt es Reichtum ohne Armut?
    \end{itemize}

    Links zum Appendix, wie Leute über Ungleichheit denken
\end{frame}

%------------------------------------------------
\section{Ceterum Censeo}
%------------------------------------------------

\begin{frame}[plain,c]
    \begin{center}
    \usebeamerfont{title}
    \color{DarkBlue}
        Wrap-up
    \end{center}
\end{frame}

\begin{frame}{Ceterum Censeo - Punkte, die mir noch wichtig sind}
    \begin{itemize}
        \item Ungleichheit kann politisch bekämpft werden. Ungleichheit war am geringsten in den 70ern, aufgrund starker Sozialstaaten.

        \pause
        \vspace{0.5cm}
        \item Wo stehst du selbst?
        \item Vor allem: Wo bist du im Vergleich zu anderen privilegiert? Denke nicht nur an Einkommen und Vermögen.
        \item Wie schlägt sich das nieder?
    \end{itemize}
\end{frame}

\begin{frame}{Ressourcen}
    \begin{itemize}
        \item World Inequality Report
        \item World Inequality Databank: https://wid.world/
    \end{itemize}
\end{frame}

\begin{frame}{Feedback}
    \begin{center}
        ?`Fragen? !`Fragen!
    \end{center}

    \vspace{0.5cm}
    \pause 
    Feedback
    \begin{itemize}
        \item Was hat euch gefallen? Was hat euch nicht gefallen? Was war so naja und könnte ich besser machen?
        \item Was nehmt ihr mit?
        \item Worüber hättet ihr gerne mehr gehört?
    \end{itemize}

    \vspace{0.5cm}
    \pause
    \begin{center}
        Kontakt: clara.himmelbuaer@wu.ac.at \\
        {\small(Please send me your memes)}
    \end{center}
\end{frame}


% \begin{frame}[allowframebreaks]{References}
%     \nocite{*} 
%     \footnotesize
%     \bibliography{ref.bib}
%     \bibliographystyle{apalike}
% \end{frame}

%------------------------------------------------
\section*{Appendix}
%------------------------------------------------

\begin{frame}[plain,c]
    \begin{center}
    \usebeamerfont{title}
    \color{DarkBlue}
        Appendix
    \end{center}
\end{frame}


\begin{frame}[label = ekverteilung_alternativ]{Nationale Einkommensverteilung (2)}
    \begin{figure}
        \centering
        \includegraphics[width=0.7\linewidth]{figures/verteilung_allgemein/akooe_einkommensverteilung.jpg}
        \label{fig:ekverteilung-alternativ}
    \end{figure}

    \hyperlink{ekverteilung_at}{\beamerskipbutton{Nationale Einkommensverteilung}}
\end{frame}

\begin{frame}[label = gender-pay-gap]{Gender-Pay-Gap}
    \begin{figure}
        \centering
        \includegraphics[width=0.7\linewidth]{figures/verteilung_allgemein/stata_gender-pay-gap.png}
        \label{fig:gender-pay-gap}
    \end{figure}

    \hyperlink{ekverteilung_at}{\beamerskipbutton{Nationale Einkommensverteilung}}
    \hyperlink{zeitverwendung}{\beamerskipbutton{Verteilung von Arbeitszeit Zeitverwendungserhebung}}
\end{frame}


\begin{frame}[label = vermoegensverteilung-alternativ]{Nationale Vermögensverteilung (2)}
    \begin{figure}
        \centering
        \includegraphics[width=0.65\linewidth]{figures/verteilung_allgemein/akooe_vermoegensverteilung.png}
        \label{fig:vermoegensverteilung-alternativ}
    \end{figure}

    \hyperlink{vermoegensverteilung_at}{\beamerskipbutton{Nationale Vermögensverteilung}}
\end{frame}

\begin{frame}[label = vermoegensverteilung-kumulativ]{Nationale Vermögensverteilung (3)}
    \begin{figure}
        \centering
        \includegraphics[width=0.8\linewidth]{figures/verteilung_allgemein/ferschli_kumulative-vermoegensverteilung.png}
        \label{fig:ferschli_vermoegensverteilung kumulativ}
    \end{figure}

    \hyperlink{vermoegensverteilung_at}{\beamerskipbutton{Nationale Vermögensverteilung}}
\end{frame}

\begin{frame}[label = zeitverwendung_erwachsene]{Verteilung von Arbeit}

    \begin{figure}
        \centering
        \includegraphics[width=0.8\linewidth]{figures/verteilung_allgemein/zve_arbeitszeit_25-59.png}
        \label{fig:zve_verteilung-arbeit_25-59}
    \end{figure}

    \hyperlink{zeitverwendung}{\beamerskipbutton{Verteilung von Arbeitszeit Zeitverwendungserhebung}}
\end{frame}


\begin{frame}[label = lobbying]{Lobbying}
    \begin{columns}

        \begin{column}{0.7\textwidth}
            \begin{itemize}
                \item Lobbying dient auch der Weitergabe von Informationen zwischen Politik und Verbänden, Unternehmen, Interessensgruppen
                \item Maßnahmen für besseres Lobbying: Mehr Transparenz und Kontrolle
                \begin{itemize}
                    \item Einsicht ins Lobbyisten-Register
                    \item Offenlegungspflichten für Lobbyist*innen und Politiker*innen
                    \item Cooling-off Phase für Politiker*innen
                    \item Keine Schlupflöcher, Gleichstellung aller Organisationen
                    \item Sanktionen bei schädlichem Lobbying
                \end{itemize}
            \end{itemize}
        \end{column}
        
        \begin{column}{0.3\textwidth}
            \begin{figure}
                \centering
                \includegraphics[width=1\linewidth]{figures/demokratie/lobbycontrol_huawei.png}
                \label{fig:lobbycontrol_huawei}
            \end{figure}
            \vspace{-1cm}
            \begin{figure}
                \centering
                \includegraphics[width=1\linewidth]{figures/demokratie/spiegel_veggie.png}
                \label{fig:spiegel_veggie}
            \end{figure}
        \end{column}

        
    \end{columns}

    \hyperlink{demokratie_einflussnahme}{\beamerskipbutton{Ungleichheit und Demokratie: Einflussnahme}}
\end{frame}

\begin{frame}[label = parteispenden]{Politische Einflussnahme über Parteispenden}

    \begin{columns}
        \begin{column}{0.5\textwidth}
            \begin{figure}
                \centering
                \includegraphics[width=1\linewidth]{figures/demokratie/oevp-spenden-2017.png}
                \label{fig:derstandard-oevpspenden}
            \end{figure}
        \end{column}

        \begin{column}{0.5\textwidth}
            \begin{figure}
                \centering
                \includegraphics[width=1\linewidth]{figures/demokratie/parteispenden-2012-17.png}
                \label{fig:apa_parteispenden}
            \end{figure}
        \end{column}
    \end{columns}
    
    \hyperlink{demokratie_einflussnahme}{\beamerskipbutton{Ungleichheit und Demokratie: Einflussnahme}}
\end{frame}

\begin{frame}[label = ueberrepraesentation]{Überrepräsentation von Millionär*innen in Politik}

    \begin{columns}
        \begin{column}{0.5\textwidth}
            \begin{figure}
                \centering
                \includegraphics[width=1\linewidth]{figures/demokratie/millionaires_congress-usa_own-piechart.png}
                \label{fig:piechart_millionaires}
            \end{figure}
        \end{column}

        \begin{column}{0.5\textwidth}
            \begin{figure}
                \centering
                \includegraphics[width=1\linewidth]{figures/demokratie/vermoegen_politiker-deutschland.png}
                \label{fig:vermoegen-deutsche-politiker}
            \end{figure}
        \end{column}
    \end{columns}
    
    \hyperlink{demokratie_einflussnahme}{\beamerskipbutton{Ungleichheit und Demokratie: Einflussnahme}}
\end{frame}

\begin{frame}[label = medien]{Medienlandschaft in Österreich}
    \begin{figure}
        \centering
        \includegraphics[width=0.55\linewidth]{figures/demokratie/medieneigentum-oesterreich.jpg}
        \label{fig:medienlandschaft-oesterreich}
    \end{figure}
    
     \hyperlink{demokratie_einflussnahme}{\beamerskipbutton{Ungleichheit und Demokratie: Einflussnahme}}
\end{frame}

\begin{frame}[label = meme-korruption]{Lobbying und Korruption}
    \begin{figure}
        \centering
        \includegraphics[width=0.6\linewidth]{figures/memes/lobbying.jpg}
        \label{fig:meme-korruption}
    \end{figure}
    \hyperlink{demokratie_einflussnahme}{\beamerskipbutton{Ungleichheit und Demokratie: Einflussnahme}}
\end{frame}

\begin{frame}[label = nichtwahlberechtigte]{Armut und Nichtwahlberechtigung (1)}
    \begin{figure}
        \centering
        \includegraphics[width=0.7\linewidth]{figures/demokratie/silc22_povmd60-stb_1.png}
        \caption{Datenbasis: EU-SILC 2022; eigene Darstellung.}
´       \label{fig:silc22_povmd60-stb_1}
    \end{figure}
    \vspace{-1cm}    
    \hyperlink{demokratie_teilhabe}{\beamerskipbutton{Ungleichheit und Demokratie: Teilhabe}}
\end{frame}

\begin{frame}{Armut und Nichtwahlberechtigung (2)}
    \begin{figure}
        \centering
        \includegraphics[width=0.7\linewidth]{figures/demokratie/silc22_povmd60-stb_2.png}
        \caption{Datenbasis: EU-SILC 2022; eigene Darstellung.}
´       \label{fig:silc22_povmd60-stb_2}
    \end{figure}
    \vspace{-1cm}
    \hyperlink{demokratie_teilhabe}{\beamerskipbutton{Ungleichheit und Demokratie: Teilhabe}}
\end{frame}


\begin{frame}[label = percapita_gdp-co2]{CO2 Ausstoß und BIP pro Kopf (Länderebene)}
    \begin{figure}
        \centering
        \includegraphics[width=0.6\linewidth]{figures/klimawandel/worldbank_emissions-gdp-per-capita.png}
        \label{fig:worldbank_emissions-gdp-per-capita}
    \end{figure}

    \hyperlink{klimawandel}{\beamerskipbutton{Klimawandel}}
\end{frame}

\begin{frame}[label = co2-einkommensverteilung]{Reiche Menschen emittieren mehr}
    \begin{figure}
        \centering
        \includegraphics[width=0.8\linewidth]{figures/klimawandel/wir_co2-emissions_inome.png}
        \label{fig:wir_co2-emissions_income}
    \end{figure}

    \hyperlink{klimawandel}{\beamerskipbutton{Klimawandel}}
\end{frame}


\begin{frame}[label = climate-risk]{Climate Risk Index}
    \begin{figure}
        \centering
        \includegraphics[width=0.8\linewidth]{figures/klimawandel/germanwatch_climate-risk-indicator.png}
        \label{fig:germanwatch_climate-risk}
    \end{figure}

    \hyperlink{klimawandel}{\beamerskipbutton{Klimawandel}}    
\end{frame}

\begin{frame}[label = memes-klimawandel]{Klimawandel-Memes}
    \begin{columns}
        \begin{column}{0.5\textwidth}
            \begin{figure}
                \centering
                \includegraphics[width=1\linewidth]{figures/demokratie/millionaires_congress-usa_own-piechart.png}
                \label{fig:piechart_millionaires}
            \end{figure}
        \end{column}

        \begin{column}{0.5\textwidth}
            \begin{figure}
                \centering
                \includegraphics[width=1\linewidth]{figures/demokratie/vermoegen_politiker-deutschland.png}
                \label{fig:vermoegen-deutsche-politiker}
            \end{figure}
        \end{column}
    \end{columns}
    
    \hyperlink{klimawandel}{\beamerskipbutton{Klimawandel}}       
\end{frame}

\begin{frame}{Wie denken Leute über Ungleichheit?}
    
\end{frame}

\begin{frame}{Messung von Ungleichheit}
    
\end{frame}

\end{document}